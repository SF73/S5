%!TEX encoding = UTF-8 Unicode
\documentclass{scrreprt}
\usepackage[utf8]{inputenc}
\usepackage[T1]{fontenc}
\usepackage{lmodern}
\usepackage[french]{babel}
\usepackage{tikz}
\usepackage{mathtools}
\usepackage{amsmath}
\usepackage{pgfplots}
\usetikzlibrary{plotmarks}
\title{}
\author{Sylvain}
\begin{document}
\maketitle
constante des gaz parfaits\newline
\[ R=8.314 \]
\[ PV=P\cdot V\cdot 0.1 \]
\[ InvV=\frac{1}{V}\cdot 1000000 \]
\[ Ppascal=P \]
\[ B0=\frac{A}{C} \]
Utilisation d'un paramètre de modélisation (donc susceptible de disparaître)\newline
\[ n=\frac{C}{R\cdot T} \]
Utilisation d'un paramètre de modélisation (donc susceptible de disparaître)\newline
\paragraph{Modélisation}
\[ PV=A\cdot InvV+B \]
\paragraph{N\degres1 }
\paragraph{N\degres2 }
Ecart-type sur PV=25,45 mJ\newline
Résultat d'un réglage manuel\newline
des paramètres. Pour optimiser,\newline
cliquer sur ajuster\newline
\paragraph{N\degres3 }
Ecart-type sur PV=26,93 mJ\newline
Résultat d'un réglage manuel\newline
des paramètres. Pour optimiser,\newline
cliquer sur ajuster\newline
\paragraph{N\degres4 }
Ecart-type sur PV=40,49 mJ\newline
Résultat d'un réglage manuel\newline
des paramètres. Pour optimiser,\newline
cliquer sur ajuster\newline
\paragraph{N\degres5 }
Ecart-type sur PV=38,05 mJ\newline
Résultat d'un réglage manuel\newline
des paramètres. Pour optimiser,\newline
cliquer sur ajuster\newline
\paragraph{N\degres6 }
Ecart-type sur PV=59,62 mJ\newline
Résultat d'un réglage manuel\newline
des paramètres. Pour optimiser,\newline
cliquer sur ajuster\newline
\paragraph{N\degres7 }
Ecart-type sur PV=23,07 mJ\newline
Intervalle de confiance à 95\%\newline
A=(-2 $\pm$0)10${^-6}$ m${^5}$.kg.s${^-2}$\newline
B=(4,86 $\pm$0,08)J\newline
\paragraph{N\degres8 }

\begin{tikzpicture}
\begin{axis}[
height=8cm,width=12cm
,axis x line=bottom,axis y line=left
,xmin=0.06,xmax=5.19
,ymin=0.082638592,ymax=4.462206208
,title={N\degres1 }
,xlabel={InvV(cm${^-3}$)}
,ylabel={PV(J)}
]
\addplot[draw=red
,mark=x
] file
 {viriel10.txt};
\addplot[draw=red
,mark=x
] file
 {viriel11.txt};
\addplot[draw=red
,mark=x
] file
 {viriel12.txt};
\addplot[draw=red
,mark=x
] file
 {viriel13.txt};
\addplot[draw=red
,mark=x
] file
 {viriel14.txt};
\addplot[draw=red
,mark=x
] file
 {viriel15.txt};
\addplot[draw=red
,mark=x
] file
 {viriel16.txt};
\addplot[draw=red
,mark=x
] file
 {viriel17.txt};
\end{axis}
\end{tikzpicture}


\begin{tikzpicture}
\begin{axis}[
height=8cm,width=12cm
,axis x line=bottom,axis y line=left
,xmin=0,xmax=5.2
,ymin=0,ymax=4.472
,grid=major
,title={N\degres1 }
,xlabel={InvV(cm${^-3}$)}
,ylabel={PV(J)}
]
\addplot[draw=red
,only marks
,mark=x
] file
 {viriel20.txt};
\addplot[draw=red
,mark=none,smooth
] file
 {viriel22.txt};
\addplot[draw=red
,mark=none,smooth
] file
 {viriel23.txt};
\addplot[draw=red
,only marks
,mark=x
] file
 {viriel24.txt};
\addplot[draw=red
,mark=none,smooth
] file
 {viriel26.txt};
\addplot[draw=red
,mark=none,smooth
] file
 {viriel27.txt};
\addplot[draw=red
,only marks
,mark=x
] file
 {viriel28.txt};
\addplot[draw=red
,mark=none,smooth
] file
 {viriel210.txt};
\addplot[draw=red
,mark=none,smooth
] file
 {viriel211.txt};
\addplot[draw=red
,only marks
,mark=x
] file
 {viriel212.txt};
\addplot[draw=red
,mark=none,smooth
] file
 {viriel214.txt};
\addplot[draw=red
,mark=none,smooth
] file
 {viriel215.txt};
\addplot[draw=red
,only marks
,mark=x
] file
 {viriel216.txt};
\addplot[draw=red
,mark=none,smooth
] file
 {viriel218.txt};
\addplot[draw=red
,mark=none,smooth
] file
 {viriel219.txt};
\addplot[draw=red
,only marks
,mark=x
] file
 {viriel220.txt};
\addplot[draw=red
,mark=none,smooth
] file
 {viriel222.txt};
\addplot[draw=red
,mark=none,smooth
] file
 {viriel223.txt};
\addplot[draw=red
,only marks
,mark=x
] file
 {viriel224.txt};
\addplot[draw=red
,mark=none,smooth
] file
 {viriel226.txt};
\addplot[draw=red
,only marks
,mark=x
] file
 {viriel227.txt};
\end{axis}
\end{tikzpicture}

\end{document}
